\chapter{Zusammenfassung}
Im Laufe der Arbeit sind einige Probleme aufgetreten, die zwar nicht unmöglich zu lösen waren, aber trotzdem nach zeitintensiven Lösungen verlangten. Anfänglich war der Bad-Designer, wegen eines Missverständnisses, ein Konfigurator. Erst nach dem zweiten Gespräch mit dem Auftraggeber wurde der Irrtum ausgebessert. Dank der Flexibilität von Three.js [\ref{sec:three.js}] ist die Umstellung auf die aktuelle Version schnell vorangegangen. Dadurch dass die Dateien mit den Modellen von den Badezimmermodulen immer größer wurden, litt die Performance beim Laden der Website darunter. Dies wurde durch Komprimieren und Umstieg auf einen effizienteren Dateityp behoben, des Weiteren werden die Objekte lokal im Cache gespeichert sobald die Webseite einmal geladen wurde, dass führt noch einmal zu einer Verbesserung der Ladezeit. Den endgültigen Perfomance Schub brachte der Dateityp GLB [\ref{sec:GLB}] und der Draco-Loader [\ref{dracoloader}]. Dadurch dass, der Bad-Designer sowohl als Web-Anwendung als auch als Electron Anwendung [\ref{sec:Electron}] angeboten wird, kann man das Programm lokal und ohne Internet benutzen um eine gewisse Unabhängigkeit zu schaffen.Für das Deployment der Diplomarbeit kam Docker [\ref{sec:Docker}] ins Spiel, durch diese Containervirtualisierung kann die Applikation in kürzester Zeit zur Verfügung gestellt werden. \\
Trotz aller Hürden ist die Diplomarbeit gelungen und der Auftraggeber, \\die {\projectpartner}, mit dem Ergebnis zufrieden. Die Ziele die gesetzt wurden, wurden auch erreicht.
