\chapter{Lizenzen}
\section{MIT-Lizenz}\label{sec:MIT}
Die freizügige Open-Source-Lizenz der amerikanischen Universität Massachusetts Institute of Technology erlaubt die Wiederverwendung von Software dessen Code sowohl frei und nicht frei einsehbar ist. Frameworks wie jQuery ~\cite{jQueryLicense}, Node.js ~\cite{NodeJsLicense} und .NET Core ~\cite{NetCoreLicense} werden unter der 1988 veröffentlichten Lizenz zur Verfügung gestellt.

\section*{Inhalt der Lizenz}
\\ \cite{MITLicense} \\
Bei der MIT-Lizenz handelt es sich um eine der duldsamsten Open-Source-Lizenzen, weshalb sie kaum Begrenzungen oder Pflichten für Nutzer enthält. Hiermit wird gebührenfrei die Erlaubnis erteilt, ohne Beschränkung mit der Software zu handeln, einschließlich der Rechte zur Benutzung, zum Vervielfältigen, Umgestalten, Zusammenführen, Publizieren, Verteilen, Unterlizenzieren und/oder Verkaufen von Kopien der Software, und Personen, denen die Software zur Verfügung gestellt wird, dies unter den untenstehenden Bedingungen zu gestatten.
\section*{Copyleft}
Die MIT-Lizenz enthält keine Copyleft-Klausel. Das bedeutet, der Benützer kann für die von ihm weiterentwickelten Softwareteile eine Lizenz seiner Wahl verwenden. Dabei hat er die Wahl, ob er seine Weiterentwicklung als proprietäre Software oder als Open Source Software lizenziert.  Der unveränderte Teil der Software verbleibt dabei aber weiterhin unter der ursprünglichen MIT-Lizenz.

\newpage


\section{BSD-Lizenz}\label{sec:BSD}
Die von der amerikanischen Universität University of California, Berkeley stammende Lizenz umfasst eine Gruppe von Open-Source-Lizenzen. Ähnlich zur MIT-Lizenz ist sie freizügig, das Akronym BSD steht für Berkeley Software Distribution.

\section*{Inhalt der Lizenz}
\\ \cite{BSDLicense}\\
Software die unter der BSD-Lizenz veröffentlicht wurde darf frei verwendet werden. Konkret bedeutet das, dass es erlaubt ist die Software zu kopieren, ändern und zu verbreiten aber der Copyright Vermerk nicht entfernt werden darf. Damit wird gewährleistet das der Entwickler vom ursprünglichen Programm gewürdigt wird. Durch diese Bedingungen können Softwareprodukte auch kommerziell gehandelt werden, wenn sie auf Technologien aufbauen, die unter der BSD-Lizenz zu freien Verfügung gestellt wurden.

\section*{Copyleft}
Das besondere an dieser Lizenz ist, dass es unter gewissen Umständen kein Copyleft enthält. Beispielweise muss also ein Programmierer, der eine Software die unter der BSD-Lizenz steht, verändert und anschließend binär verbreitet nicht den Quellcode mitveröffentlichen. Jedoch muss er das Programm in nichtkompilierter oder kompilierter Form weiterhin unter der BSD-Lizenz veröffentlichen samt Lizenztext.

\newpage
\section{GNU General Public License}\label{sec:GNU}
Die General Public License, aus dem englischen übersetzt etwa\\ \textit{allgemeine Veröffentlichungserlaubnis} ,ist die meistbenutzte Softwarelizenz, die es gewährt eine Software auszuführen, manipulieren und zu verbreiten. Programme, die unter dieser veröffentlicht werden als freie Software bezeichnet. Das bedeutet das die Freiheit der Nutzer im Zentrum steht und ihnen gleichzeitig die Nutzungsrechte nicht eingeschränkt werden. Die Erstfassung stammt von Richard Stallman von der Free Software Foundation, die über das Copyright des Lizenztextes verfügen. Die Lizenz wird in unregelmäßigen Abständen aktualisiert.

\section*{Inhalt der Lizenz}
\\ \cite{GNULicense} \\
Die tolerante Lizenz lässt zu das unter ihr die Software sowohl kommerziell als auch kostenlos angeboten werden darf. Jedoch muss dabei wegen des Copylefts die Änderungen und der Quellcode dem Endnutzer offen gelegt werden. 

\section*{Copyleft}
Änderung und Abwandlungen von GPL lizensierten Arbeiten dürfen nur unter derselben Lizenz vertrieben werden.

