\chapter{Einleitung}
\section{Ausgangslage}
%neuer Text
Das in Leonding ansässige Unternehmen {\projectpartner} designt für Hotels, Bars, Restaurants das Interieur um für eine angenehme Atmosphäre zu sorgen. Vor allem für Badezimmer werden die Dienste der {\projectpartner} in Anspruch genommen, deshalb haben Sie das modulare Bad entwickelt. SanMod-Bäder bieten individuelle, freistehende Sanitärmodule für einen flexiblen und zeitsparenden Einbau in Gebäuden. Zurzeit werden die Bäder dem Kunden mit einem Video präsentiert. Diese Methode bringt aber Nachteile mit sich, da die Bäder immer nur aus einer Perspektive gezeigt werden, die Modelle unflexibel sind, nicht interaktiv und Kundenwünsche nicht sofort umgesetzt werden können. Diese Probleme werden durch die Applikation behoben.

\section{Zielsetzung}
Das Ziel ist es, die Planung zu erleichtern und gegebenenfalls Änderungen sofort im Kundengespräch umzusetzen. Durch diese virtuelle Stütze können sich alle Beteiligten das fertige Bad besser vorstellen und mögliche\\ Missverständnisse sofort klären. Kundengespräche werden dadurch aufgewertet und effizienter.

\section{Aufbau der Diplomarbeit}
Die Diplomarbeit ist in drei Teile gegliedert. 
Der erste Teil besteht aus den verwendeten Technologien.
Der zweite Teil geht näher auf die Grundlagen und Methoden der Arbeit ein.
Der dritte und letzte Teil behandelt die Umsetzung und Realisierung des Projektes.
Außerdem wird hier auf die Systemarchitektur eingegangen und die Implementierung so wie
der Release der Diplomarbeit.