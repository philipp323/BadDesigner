\chapter{Dateien}

\section{.FBX}
~\cite{FBX_01} ~\cite{FBX_02}
\section*{Erläuterung}
3D-Objekte, 2d-Objekte mit Objekthöhe, Lichtquellen, Kameras und Materialien werden in dem vom AutoDesk entwickelten Dateiformat gespeichert. Das Universalformat, das bedeutet, dass es von diversen Programmen geöffnet werden kann, enthält unteranderem auch Übergänge, Video- und Audiodateien. 

\section*{Funktionalität}
Dadurch das FBX-Dateien die volle Funktionalität und Wiedergabetreue der Originaldatei beibehalten, ist es zum Beispiel möglich mit Cinema 4D, 3D Studio Max und PowerAnimator den Projektentwurf zu bearbeiten obwohl er davor in AutoCAD erstellt wurde. 

\section*{Verwendung}

\clearpage
\newpage

\section{.OBJ}
~\cite{OBJ_01}~\cite{OBJ_02}
\section*{Erläuterung}
Das offene Dateiformat .obj dient zum Speichern von dreidimensionalen geometrischen Formen. Das von Wavefront Technologies erstmals 1989 veröffentlichte Format gilt als universelles Dateiformat für Dreidimensionale-Grafikbearbeitungsprogramme. Daher eignet es sich für sowohl plattform- als auch programmübergreifende Weitergabe von Modellen.

\section*{Funktionalität}
Das OBJ-Format speichert lediglich die geometrischen Eigenschaften wie die Koordinaten der Ecken, Texturen, Normalen, Flächen und Glättungen. Die Oberflächenschattierungen können durch Referenzen auf .mtl-Dateien implementiert werden.

\section*{Verwendung}

\clearpage
\newpage

\section{.GLB}
~\cite{GLB_01}
\section*{Erläuterung}
In GLB-Dateien werden Repräsentationen von dreidimensionalen Modellen im GL Transmission Format, binär abgespeichert. Das GL Transmission Format speichert dreidimensionale Objekte im JSON-Format ab. Des Weiteren werden Kameras, Materialien, Animationen und Meshes binär in der GLB-Datei abgespeichert. 
\section*{Funktionalität}
Durch die kompakten Dateigrößen und das JSON-Format können so komplette dreidimensionale Szenen Repräsentationen schnell und effizient geöffnet werden. Dies ist wichtig für die zügige Darstellung der Badezimmer Module im Bad-Designer.


\section*{Verwendung}




